\documentclass[conference, compsoc]{IEEEtran}
\IEEEoverridecommandlockouts
% The preceding line is only needed to identify funding in the first footnote. If that is unneeded, please comment it out.
\usepackage{cite}
\usepackage{amsmath,amssymb,amsfonts}
\usepackage{algorithmic}
\usepackage{graphicx}
\usepackage{textcomp}
\usepackage{xcolor}
\usepackage{hyperref}
\def\BibTeX{{\rm B\kern-.05em{\sc i\kern-.025em b}\kern-.08em
    T\kern-.1667em\lower.7ex\hbox{E}\kern-.125emX}}
\begin{document}

\title{Generating Binomial and Truncated Normal Random Variables}
\author{
    \IEEEauthorblockN{Michele Veronesi\IEEEauthorrefmark{0}}
    \IEEEauthorblockA{\IEEEauthorrefmark{0}Politecnico di Torino, Turin, Italy \\
    \href{mailto:s296599@studenti.polito.it}{s296599@studenti.polito.it}}
}

\maketitle

\begin{abstract}
In this document we report the work done to create two generators:
the first simulate the behaviour of a binomial random variable,
the second the one of a truncated normal distribution.
While the former present three possible methods for generating instances
(convolutional, inverse-transform, geometric), the latter is implemented using only the acceptance/rejection technique.
Results are... % TO-DO
The code used during our experiments is available at \cite{code_repository}
\end{abstract}


\section{Binomial distribution generator}


\section{Truncated normal distribution generator}


\bibliography{bibliography}
\bibliographystyle{IEEEtran}


\end{document}
